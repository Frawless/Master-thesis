% (c) Jakub Stejskal
% Master Thesis
% Performance Testing and Analysis of Qpid-Dispatch Router
% Chapter 8

\chapter{Summary}
\label{Summary}
In this report we described the fundamentals of performance testing, common performance metrics and bugs and selected related tools. Further, we introduced the architecture and functionality of Messaging Performance Tool (MPT), and proposed its extension necessary to performance testing of Qpid-Dispatch tool. Moreover, we designed and implemented the Topology Generator tool, which is going to be used for semi-automatic topology configuration generation, which will significantly simplify the testing phase. The source code of the topology generator can be found in Appendix \ref{AP:Topology Generator Source Code}.

In the follow-up work, the core of the thesis\,---\,a module for Qpid-Dispatch performance testing\,---\,will be implemented, integrated into MPT and the architecture and functionality of Topology Generator will be refined to allow precise performance testing of Qpid-Dispatch tool together with fine data analysis. The work will be concluded with experimental evaluation based on implemented modules and tools.

The results of this measurements was described and published in the paper for Excel@FIT\footnote{Excel@FIT\,---\,IT conference for students and theirs work \url{http://excel.fit.vutbr.cz/}} conference.

%During term project is collected relative information about performance testing. This information are processed and recorded in this work in the Chaper \ref{Fundamentals of Software Performance Testing}. I~also get introduction into functionality of Messaging Performance Tool which is going to be upgraded and used for performance testing of Qpid-Dispatch. In the Chaper \ref{Analysis and Design} I~summarized design of necessary upgrades for MPT and design of Topology Generator which going to be used for semi-automatic topology configuration generation and together with automatic deployment on testing nodes. Topology is currently implemented, tested and published under version 0.1.10. Current version is able to generate any topology by manner as is described in the Chapter \ref{Analysis and Design}. Link to public source-code is available in \ref{Topology Generator Source Code}.

%Next step of this thesis will lead to implement module for Qpid-Dispatch performance testing, integrated it into MPT and improved Topology Generator.After this, I~will be ready for proper performance testing of Qpid-Dispatch with data analysis.

%=========================================================================
