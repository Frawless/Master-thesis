% (c) Jakub Stejskal
% Master Thesis
% Performance Testing and Analysis of Qpid-Dispatch Router
% Chapter 1

% Zajimavy blog
% http://brendangregg.com/

%!TeX spellcheck = en-US

\chapter{Introduction}
\label{Introduction}
Good application performance is one of the main goals during the software development. But what makes software performance so important? Software reliability has to be guaranteed by the owner, but with undesirable performance there could still be a lot of issues, which can badly influence the software behavior. And this can cause a significant outflow of the consumers, and even brand destruction, financial damage, or loss of trust. These few reasons should be enough to do a proper performance testing before every software release, especially for large projects where industries have to guarantee certain level of software behavior and they would not be able to assure it with insufficient performance testing. Great emphasis on software performance is, in particular, in space programs, medical facilities, army systems, or energy distribution systems. In these fields it is necessary to ensure proper application behavior for a long time under a high load and without any unexpected behavior such as high response time, frequent delays, or timeouts, because every failure is paid dearly.

Nowadays every developer should try to use well established frameworks which can make theirs work easier. Frameworks already handle complex underlying issues such as security, performance, or code clarity. This way developers can invest more time in the actual functionality and meet the application requirements, since frameworks are usually optimized for one particular job. In the past every developer had to spent significant portion of development time tuning the performance which naturally led to spending more time and money for software development. But not everyone has enough knowledge of performance testing and this makes performance analysis and optimization even more difficult. This leads to a~need for specialized performance tools which can provide more sophisticated information, however, useful tools are usually proprietary or are too expensive.

A~very important part of the performance analysis is the right choice of so called \emph{key performance indicators} (KPIs) \cite{Molyneaux:TAoAPT} and effective interpretation of the results. The right choice of KPIs allows faster detection of performance problems and help developers with fixes and meeting the \emph{performance standards} \cite{Molyneaux:TAoAPT} set up by application owner or customer in time before the release.

In general an application performance is important. However, smooth network application or hardware performance became much more demanded nowadays, since most of the communication is performed via the Internet.  Obviously when you make a payment in your internet banking you definitely want to have a stable connection to your bank's website without any delay. Network stability is significantly influenced by network components like routers and switches and hence their performance should be under the utmost case. We refer to network performance testing as measurement of network service quality which is directly influenced by \emph{bandwidth, throughput, latency}, etc.

For performance testing of particular network messaging systems developed by \emph{Red~Hat~Inc.} there is an existing solution\,---Messaging Performance Tool (MPT) called \emph{Maestro} \cite{ORPISKE:MSGPT}. MPT is specialized for the performance testing of \emph{AMQ Broker} (Message Broker) \cite{RH:Broker}\,---\,network application level software cooperating with \emph{Qpid-dispatch service} \cite{RH:Interconnect} in the network as the message distributor. Unfortunately, the current version of Maestro does not support performance testing of enough components like the Router component, Qpid-dispatch. In this work we will focus on this particular short coming and develop a worthy solution allowing proper performance testing of the Qpid-dispatch service.

This thesis is structured as follows. First, we define fundamentals of performance testing in Chapter \ref{Fundamentals of Software Performance Testing}. The rest of the thesis focuses on performance testing and analysis of Qpid-dispatch, an application level router designed by Red~Hat~Inc. Qpid-dispatch performance testing is based on Maestro described in Chapter \ref{Messaging Performance Tool}. Description includes \emph{measurement process} and \emph{measured data description and evaluation}.The main goal of the thesis is to analyze Maestro and design module for the Qpid-dispatch performance testing as described in Chapter \ref{Analysis and Design} together with used protocols and \emph{Automatic Topology Generator} for semi-automated network generation and deployment. Used technologies, tools and  implementation processes of each component are described in Chapter \ref{Implementation}. The most important part of the thesis is Chapter \ref{Experimental Evaluation}, containing the data gathering  from routers located in different types of topology, data evaluation and representation which leads to conclusion about performance of Qpid-dispatch. Finally, Chapter \ref{Conclusion} summarizes the thesis and proposes ideas for future use of developed tool.

% https://www.seguetech.com/what-is-software-performance-testing/
