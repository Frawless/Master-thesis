% Tento soubor nahraďte vlastním souborem s přílohami (nadpisy níže jsou pouze pro příklad)
% This file should be replaced with your file with an appendices (headings below are examples only)

% Umístění obsahu paměťového média do příloh je vhodné konzultovat s vedoucím
% Placing of table of contents of the memory media here should be consulted with a supervisor
%\chapter{Obsah přiloženého paměťového média}

%\chapter{Manuál}
\chapter{The Maestro Protocol}

\section{The Maestro Commands}
\label{AP:commands}
\todo{\url{https://github.com/orpiske/msg-perf-tool/tree/master/doc/maestro/protocol}}

\chapter{Topology Generator} % Configuration file

\section{Inventory}
\label{AP:Inventory}
An example of Inventory file for Topology Generator and Ansible deployment scripts.

\begin{verbatim}
[clients]
sender ansible_host=10.0.0.1					
receiver ansible_host=10.0.0.2

[routers]
router1 ansible_host=10.0.0.3
router2 ansible_host=10.0.0.4

[brokers]
broker1 ansible_host=10.0.0.5

[nodes:children]
brokers
clients
routers
\end{verbatim}

\section{Graph Metadata}
\label{AP:Graph Metadata}
Example of graph metadata file for Topology Generator. Generator will generate graph with 2 routers and 3 brokers where routers are connected together and each broker is connected to one router.

\begin{verbatim}
---
directed: false
graph: {}
nodes:
- type: router						%node type
  id: router1						%node name
- type: router
  id: router2
- type: broker
  id: broker1
- type: broker
  id: broker2
links:
- source: router2					%source node for link
  target: router1					%target node for link
- source: router2
  target: broker2
- source: router1
  target: broker1
multigraph: false
\end{verbatim}

\section{Topology Generator Output}
\label{AP:Topology Generator Output}
Example of Topology Generator output in YAML format. This output is for two directly connected routers.

\begin{verbatim}
---
confs:
- machine: router1
  router:
  - id: router1
    mode: standalone
  listener:
  - host: 0.0.0.0
    role: inter-router
    port: 6000
  - host: 0.0.0.0
    authenticatePeer: 'no'
    role: normal
    port: 5000
    saslMechanisms: ANONYMOUS
  connector:
  - host: router2
    role: inter-router
    port: 6001
  address:
  - prefix: closest
    distribution: closest
  - prefix: multicast
    distribution: multicast
  - prefix: unicast
    distribution: closest
- machine: router2
  router:
  - id: router2
    mode: standalone
  listener:
  - host: 0.0.0.0
    role: inter-router
    port: 6001
  - host: 0.0.0.0
    authenticatePeer: 'no'
    role: normal
    port: 5001
    saslMechanisms: ANONYMOUS
  connector:
  - host: router1
    role: inter-router
    port: 6000
  address:
  - prefix: closest
    distribution: closest
  - prefix: multicast
    distribution: multicast
  - prefix: unicast
    distribution: closest

\end{verbatim}

\section{Qpid-Dispatch Configuration File Template}
\label{AP:Qpid-Dispatch Configuration File Template}
Template for configuration files for current version of Qpid-Dispatch is available at \url{https://github.com/rh-messaging-qe/ansible-qpid-dispatch/blob/master/test/files/templates/qdrouterd-roland.conf.j2}.

\section{Topology Generator Source Code}
\label{AP:Topology Generator Source Code}
Complete source code of Topology Generator is available at \url{https://github.com/rh-messaging-qe/iqa-topology-generator}.

%\chapter{RelaxNG Schéma konfiguračního souboru} % Scheme of RelaxNG configuration file

%\chapter{Plakát} % poster
