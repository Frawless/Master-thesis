%=========================================================================
% (c) Michal Bidlo, Bohuslav Křena, 2008

\chapter{Introduction}

\chapter{Fundamentals of Software Performance Testing}

\section{Performance Testing Process}

\section{Performance Issues}

\subsection{Response Time}

\subsection{Traffic Spikes}

\subsection{Performance Degradation}

\section{Different Types of Performance Testing}
% % http://www.wmrichards.com/high_performance_messaging.pdf 2017/10/18

\subsection*{Robustness Testing}
% Soak

\subsection*{Stress Testing}

\subsection*{Load Testing}

\subsection*{Bench-marking}

\section{Performance Metrics}
% https://loadstorm.com/load-testing-metrics/ 2017/10/18

\subsection{Response Time}

\subsection{Requests per Second}

\subsection{Resource Usage}

\subsection{Throughput}

\subsection{Error Rate}

\chapter{Messaging Performance Tool}
% https://github.com/orpiske/msg-perf-tool
\section{Measures Process}

\section{Testing Metrics}

\section{Gathered Data and Their Evaluation}

\chapter{Design}

\section{Qpid-Dispatch Router}

\section{Usable Protocols}
AMQP, MQTT - possibly?

\section{Automatic Topology Generator}

\subsection{Network Components}

\subsection{Structure of Input and Output}

\subsection{Topology Creation}

\section{Qpid-Dispatch Performance Module}

\subsection{TODO - more subsections about module}

\section{Performance and Testing Metrics of Qpid-Dispatch Performance Module}

\section{Gathered Data Evaluation}

\chapter{Implementation}

\section{Used Technologies}

\subsection{Ansible}

\subsection{Docker}
Using for testing Ansible roles (remove?)

\subsection{Python}

\subsection{C}

\section{Topology Generator}

\subsection{Template Generator}

\subsection{Generation of Variables}

\subsection{Configuration Files Generation and Deployment}

\section{Qpid-Dispatch Performance Module}

\subsection{TODO - more subsections about implementation}

\chapter{Testing and Experiments}

\section{Performance Testing on Various Generated Topology}

\section{Testing results}

\chapter{Conclusion}

%=========================================================================
