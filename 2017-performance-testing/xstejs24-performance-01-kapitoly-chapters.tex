%=========================================================================
% (c) Michal Bidlo, Bohuslav Křena, 2008

% Zajimavy blog
% http://brendangregg.com/


\chapter{Introduction}
\label{Introduction}
Good application performance is one of the main goals during the software development. But what makes software performance so important? Software reliability has to be guaranteed by the owner, but with undesirable performance there could be a lot of issues. Thus can badly influence software behavior. And this can cause a significant outflow of the consumers, and even brand destruction, financial damage, or loss of trust. These few reasons should be enough to do a proper performance testing before every software release, especially for large projects where industries guarantee certain level of software behavior and they would not be able to assure it with insufficient performance testing. Great emphasis on software performance is, in particular, in space programs, medical facilities, army systems or energy distribution systems. In these fields it is necessary to ensure proper application behavior for a long time under high load and without any unexpected behavior such as high response time, frequent delays or timeouts, because every failure is paid dearly. 

Nowadays every developer should try to use well established frameworks which can make theirs work easier. Frameworks handle complex underlying issues such as security, performance, and code clarity. This way developers can invest more time in the actual functionality and meet the application requirements, since frameworks are usually optimized for one particular job. In the past every developer had to spent significant portion of development time tuning performance which led to spending more time and money for software development. But not everyone has enough knowledge of performance testing and this makes performance analysis and optimization even more difficult. This leads to a need for specialized performance tools which can provide more sophisticated information, however, actually useful tools are usually proprietary and/or too expensive.

A~very important part of the performance analysis is the right choice of \emph{key performance indicators} (KPIs) \cite{Molyneaux:TAoAPT} and effective interpretation of the results. The right choice of KPIs allows faster detection of performance problems and help developers with fixes and meeting the \emph{performance standards} \cite{Molyneaux:TAoAPT} set up by application owner or customer in time before the release. 

In general an application performance is important. However, smooth network application or hardware performance became much more demanded nowadays, since most of the communication is via the Internet.  Obviously when you make a payment in your internet banking you definitely want to have a stable connection to your bank's website without any delay. Network stability is significantly influenced by network components like routers and switches and hence their performance should be under utmost case. We refer to network performance testing as measurement of network service quality which is directly influenced by \emph{bandwidth, throughput, latency}, etc. 

For performance testing of particular network messaging system developed by \emph{Red Hat Inc.} there is an existing solution\,---Messaging Performance Tool (MPT) \cite{ORPISKE:MSGPT}. MPT is specialized for the performance testing of \emph{Message Broker} (Broker) \cite{RH:Broker}\,---\,network application level software cooperating with \emph{Qpid-dispatch service} \cite{RH:Interconnect} in the network as the message distributor. Unfortunately, the current version of MPT does not support performance testing of enough component like the Router component, Qpid-dispatch. In this work we will focus on this particular short coming and develop a worthy solution allowing proper performance testing of the Qpid-dispatch service.

This thesis is structured as follows. First, we define fundamentals of performance testing in Chapter \ref{Fundamentals of Software Performance Testing}. The rest of the thesis focuses on performance testing and analysis of Qpid-dispatch, an application level router designed by Red Hat Inc. Qpid-dispatch performance testing is based on MPT described in Chapter \ref{Messaging Performance Tool}. Description includes \emph{measurement process} and \emph{measured data description and evaluation}.The main goal of the thesis is to analyze MPT and design module for the Qpid-dispatch performance testing as described in Chapter \ref{Analysis and Design} together with used protocols and \emph{Automatic Topology Generator} for semi-automated network generation and deployment. Used technologies, tools and  implementation processes of each component are described in Chapter \ref{Implementation}. The most important part of the thesis is Chapter \ref{Experimental Evaluation}, containing the data gathering  from routers located in different types of topology, data evaluation and representation which leads to conclusion about performance of Qpid-dispatch. Finally, Chapter \ref{Conclusion} summarizes the thesis and proposes ideas for future use of developed tool.

% https://www.seguetech.com/what-is-software-performance-testing/

\chapter{Fundamentals of Software Performance Testing}
\label{Fundamentals of Software Performance Testing}
Usual goal of performance testing is to ensure that the application runs reasonably fast enough to keep the attention of users, even with unexpected amount of clients using the application at the same time. But why is it so important to have the application optimized for the best speed? Simply, when your application have slow response, long load time or bad scalability, the first website which user will visit afterwards will be web of your competitor. That is the reason why speed is currently one of the most significant performance factor of common performance problems. This chapter summarizes the fundamentals of performance testing which includes common performance processes, issues, and metrics, based on knowledge available in \cite{Molyneaux:TAoAPT, Kurkova:Thesis:2017, DIN:PHD, ISTQB}.


\section{Performance Testing Process}
\label{Performance Testing Process}
The main goal of the performance testing is to ensure the following application attributes \cite{GAO:MEASURING}:

\begin{description}
	\setlength\itemsep{0em}
	\item \textbf{Reliability and Stability}\,---\,the ability of software to perform its functions in system environment under some system load for acceptable \footnote{During software development there is a document with Software Requirements Specification which specifies software metrics, including performance.} period of time,
	\item \textbf{Scalability}\,---\,the ability of software to behave properly under various types of system load and handle increasing amount of workload (such as network traffic, server load, data transfer, etc.),
	\item \textbf{Processing time and Speed}\,---\,the ability of software to react quickly without low response time during any acceptable system load,
	\item \textbf{Availability}\,---\,the ability of software to make all of its functions available during any acceptable system load.
\end{description}

Similarly to software development process, performance testing process consist of usual engineering steps ranging from requirements definition to data evaluation. These steps also includes design, implementation of performance tests and execution with data collection. The graphical representation of the performance testing process is depicted in the Figure~\ref{fig:performace_testing_process}. 

\begin{figure}[H]
  \includegraphics[width=16cm]{obrazky-figures/pyramid.pdf}
  \captionsetup{justification=centering}
  \caption{Performance Testing Process with the four most important parts and theirs individual steps based on \cite{Sharma:HP}.}
  \label{fig:performace_testing_process}
\end{figure}

The first step of performance testing process is the selection of \emph{performance requirements} for the application. In this step, testing engineer has to analyze \emph{software under test - S.U.T}, suitable performance metrics, that will model the application performance, and set performance requirements. The result should include answers to questions such as:

\begin{itemize}
	\setlength\itemsep{0em}
	\item How many end users will the application need to handle at release, after 6 months or in 1 year ?
	\item Where will these users be physically located, and how will they connect to the application?
	\item How many end users will be concurrently connected in average at release, after 6 months and 1 year?
\end{itemize}

Based on answer to these studies, the engineer should be able to select important key performance indicators for performance test cases. Some of these indicators may be \emph{response time}, \emph{stability}, \emph{scalability}, or \emph{speed}. However, there is huge amount of possible indicators so it is necessary to properly analyze the whole application and also take into consideration another needs like an error rate, system resources, etc.  Result of this phase should be a binding document with all performance requirements to be tested and, in case of detected performance degradation, such defect must be fixed with reference to this document.

The next step is to define the \emph{performance testing strategy}, corresponding to planning and design. It is extremely important to allocate enough time for S.U.T testing effectively, because, as it was mentioned in Chapter \ref{Introduction}, performance testing is not an easy task and detecting all of the possible issues of tested components is very time consuming process. Every plan should take into account the following considerations:

\begin{description}
	\setlength\itemsep{0em}
	\item \textbf{Prepare the test environment}\,---\,this step include choosing hardware machines for testing and installing the necessary software for running load injectors, tested components, etc.
	\item \textbf{Provide sufficient workload injectors}\,---\,preparing the workload injector may take few days; we usually requires a few workstations or servers to simulate real traffic.
	\item \textbf{Identify and implement use cases}\,---\,this include identification of important parts of the system which may have an impact on performance; time needed for each use case may be different because some use cases can be simple such as navigating to a web application home page, but some may be complex such as filtering specific communication.
	\item \textbf{Instrument the test environment}\,---\,install and configure the monitoring software on the test environment.
	\item \textbf{Deal with detected problems}\,---\,test can detect significant performance issues and their fix may take a long time.
\end{description}

While this process seems trivial, the opposite is true, in particular in case of network applications. Most of performance issues manifest at big workloads or high number of users, e.g. when million users are sending requests to the network device at the same time it could lead to an unacceptable device crash. Workload injectors are designated to simulate real user activity, and allows automatic analysis of performance behavior for tested application or device. Depending on the used technology, there can be a limit on the number of virtual users that can be generated by a single injector. These automated workload injectors are necessary for effective performance testing.

After describing the plan we implement and execute proposed test cases. Environment and workload injectors are ready for execution, so last step before the testing itself is the implementation of tests. Thanks to the careful planing, engineers should have enough time to implement test cases with reference to proposed design. 

Final step of performance testing process is results evaluation. Output of this step is usually technical report with all selected performance key indicators, used workload and collected data for each test case. Then follows the data evaluation with thorough analysis of degradation localization. Additionally, the report usually contains syntactical graphs which display performance metrics along the duration of test execution.

\section{Performance Issues}
\label{Performance Issues}
\emph{Performance issue} is a common label for an unexpected application or device behavior which affects its performance. Usually, those issues are hard to detect because they manifest only under certain circumstances such as high load or long application run time. In the network applications there are several particular issues that are more frequently occurring  than others. In following, we will describe selected issues in more detail.

\subsection*{Performance Degradation}
\label{Performance Degradation}
Unclean code usually leads to inefficient algorithms, application deadlocks, or memory leaks, which all can eventually cause the performance degradation. The problem is that these issues are usually detected only during the long run time of application or inability of an application to handle high load. For this kind of issues there is a performance testing method called \emph{soak testing} \cite{BUCH:4TYPES, Manzor:APTB} which is described in Section \ref{Soak Testing}. Soak test is intended to identify problems that may appear only after long period of application run-time, hence its necessary to run this type of tests during network application development. The network applications are usually need to be available for 24 hours per day. The duration of a soak test should have some correlation to the operational mode of the system under test. Following scenarios may represent performance issues detectable by soak tests:

\begin{itemize}
	\setlength\itemsep{0em}
	\item a constant degradation in response time, when the system is run over the time,
	\item any degradation in system resources that are not apparent during short runs but will surface during long run time such as free disk space, memory consumption or machines handles,
	\item a periodical process that may affect the performance of the system, but can be detected only during long run time as backup process, exporting of data to a 3rd party system, etc.,
	\item development of new features for already using components.
\end{itemize}


\subsection*{Response Time}
\label{Response Time 1}
Response time is time it takes system to accept, evaluate and respond to the user for his request e.g. HTTP request for particular website. Different actions and requests can have significantly different response time and with that provide different load on the system. For example retrieving document from web-server by its ID is considerably faster than searching for the same document by keywords. Response time is mostly measured during the \emph{load test} \cite{Manzor:APTB} of the application. Well designed test should consider different types of load on the system, various kind of requests, and different number of connected end-users at the same time. For user based systems we usually consider 3 threshold for the response time values: 

\begin{description}
	\setlength\itemsep{0em}
	\item \textbf{0.1 second}\,---\,this represent an ideal response time for the application, because user feels that system is reacting instantly and does not notice any interruptions.
	\item \textbf{1 second}\,---\,this is the highest acceptable response time when user still does not feel any interruptions, but can feel a little delay; this still represent no bad impact on the user experience.
	\item \textbf{10 second}\,---\,this is the limit after which response time become unacceptable and user will probably stop using your application.
\end{description}

However response time limits for messaging system are more strict. They could acquire values in milliseconds or less.

\todo{Prvni iterace}

\subsection*{Traffic Spikes}
As \emph{traffic spike} \cite{Kurkova:Thesis:2017, AMC:SPIKES} we can understand the sudden degradation of one of the performance metrics such as \emph{throughput, bandwidth, error rate, response time} but also resource usage such as \emph{memory usage, disk space, etc.} during the network traffic. In real network, spikes are result of high workload, e.g. caused by higher amount of users trying to concurrently use the service over the network. For example we can experience sudden traffic spike in response time after publishing new popular viral context on video servers, start of sales events, reservation of limited amount of tickets or subject registration at university.

Traffic spikes can lead to inappropriate system behavior such as \emph{long response time}, \emph{bad throughput} and \emph{limited concurrency}. To prevent the impact of traffic spikes on system performance, it is  necessary to do sophisticated infrastructure monitoring and network load analysis, in order to distinguish between normal traffic and attack on the system. Suitable method for testing of spikes is called \emph{stress testing} \cite{Manzor:APTB} and it is described in Section \ref{Types of Performance Testing} in more details. Network system should also be scalable, thus it should be able to redirect traffic to another node with same service in case of high load which can cause performance issues due inappropriate resource usage.

\begin{figure}[H]
  \centering
  \includegraphics[width=14cm]{obrazky-figures/traffic_spike.pdf}
  \caption{The graph shows amount of concurrent sessions depending on time. During to network traffic monitoring the traffic spike occurring after 5 hours from test start.}
  \label{fig:spikes}
\end{figure}

\section{Types of Performance Testing}
\label{Types of Performance Testing}
% % http://www.wmrichards.com/high_performance_messaging.pdf 2017/10/18

For performance testing there are many types of suitable test methods. Which test you should use is determined by the nature of the system, testing requirements or how much time we have left for the performance testing. The following terms are generally well known and used in practice and each of them characterizes category or suite of the tests. Their description is based on knowledge available in \cite{TuPo:TESTS, BUCH:4TYPES, Molyneaux:TAoAPT, ISTQB}.

\subsection*{Load Testing}
Load testing is a testing method which studies how the system behaves during different types of workload within acceptable time range. Basically it simulates the real-world load. During the load test we mainly focus on metric response time of the system for requests. Requests are generated by users or another systems communicating with the S.U.T. Main goal is to determine if the system can handle required workload according to performance requirements. Load test is designed to measure the response time of system transactions under normal or peak workload. When the response time of the system dramatically increases or becomes unstable, we conclude that system reached its maximum operating capacity. After successful testing, we should mark the workload requirements as fulfilling or analyze collected data and report issues to the developers. In the Figure \ref{fig:load_test} you can see the graph of load test showing workload of raising requests to the web server at the same time where the system response time does not exceed 3.5 seconds.

\begin{figure}[H]
  \centering
  \includegraphics[width=15cm]{obrazky-figures/load_testing.pdf}
  \caption{Response time of the system during the load testing depended on by load size.}
  \label{fig:load_test}
\end{figure}

The following lists show common scenarios for load testing:
\begin{itemize}
	\setlength\itemsep{0em}
	\item The system interact with multiple users at same time.
	\item The system tracking communication and analyze it.
	\item Web services and information systems.
\end{itemize}
Typical system issues covered by load testing:
\begin{itemize}
	\setlength\itemsep{0em}
	\item Concurrent users connections can eventually result into slow response time or system crash.
	\item Network systems without redundancy connections can shutdown whole network under normal defined workload.
	\item Data availability during multiple session to data server.
\end{itemize}

\subsection*{Stress Testing}
\label{Stress Testing}
Stress testing is the specific type of load testing, where we do not measure normal workload, but focus on unexpected workloads or traffic spikes. The main purpose is to study how the system behaves in extreme conditions such as enormous number of concurrent requests, using a server with much less memory or a weaker CPU, and analyze the system performance threshold. Its very useful to know performance threshold in order to know the difference between performance under normal workload and performance threshold. The following enumeration lists common stress test scenarios: 
\begin{itemize}
	\setlength\itemsep{0em}
	\item Monitor the system behavior with maximum number of users logged in at the same time.
	\item All user performing critical operations at the same time.
	\item All users accessing the same file at the same time.
	\item Hardware issues such as server in cluster down.
\end{itemize} 
Typical issues, which are covered by stress testing:
\begin{itemize}
	\setlength\itemsep{0em}
	\item \todo{todo}
\end{itemize}

When engineers finish stress testing and found the limits of the system, they also can test the system recovery after crash during finding of the system limits.

In the Figure \ref{fig:stress_test} is recorded stress testing with raising load and response time. Everything is fine until the amount of requests exceed 3000. With higher load there comes performance issues which leads to unexpected rise of the response time.

\begin{figure}[H]
  \centering
  \includegraphics[width=15cm]{obrazky-figures/stress_testing.pdf}
  \caption{Stress testing diagram capturing dependency of response time on mount of requests.}
  \label{fig:stress_test}
\end{figure}

\subsection*{Soak Testing}
\label{Soak Testing}
Soak, or stability/endurance testing refer to the method, that tries to identify problems, that may appear only after the extended period of time e.g. The system could seems stable for one week, but after some longer period, problems such as memory leaks or not enough disk space can appear. Soak tests mainly focuses on measuring of memory as performance metric. The following are common issues found by soak test:
\begin{itemize}
	\setlength\itemsep{0em}
	\item Serious memory leaks that can eventually result into the system crash.
	\item Improperly closed database connections that could starve the system.
	\item Improperly closed connections between system layers that could stall any of the system modules.
	\item Step-wise degradation that could lead to high response time and the system becomes inefficient.
\end{itemize}
Typical scenarios for use soak testing:
\begin{itemize}
	\setlength\itemsep{0em}
	\item Developed system uses multiple database connections.
	\item There is a chance for inappropriately allocated memory, and memory free.
	\item Disk space limitation for store logs.
\end{itemize}


This sort of test needs to use appropriate monitoring system to achieve high efficiency. Problems detected by soak tests are typically manifested by gradual system slowdown in response time or as a sudden lost of system availability.

\begin{figure}[H]
  \centering
  \includegraphics[width=15cm]{obrazky-figures/soak_testing.pdf}
  \caption{Soak testing with memory usage dependent on time.}
  \label{fig:soak_test}
\end{figure}

In the Figure \ref{fig:soak_test} you can see raising memory usage after period of time. The STU can handle requests but as time goes by memory usage is to high that the STU will crash. This may be caused by memory leak or an inappropriate algorithm use.

\subsection*{Smoke Testing}
Smoke testing method is inspired by similar hardware technique, when engineers checks for presource of the smoke from the device after turning the power on. Basically, its similar for software, since main goal of smoke test is to test basic functionality of the system and guarantee that the system is ready for build. However, smoke tests are testing the functionality on a surface level, so it may not be enough for deep testing of basic system functions. When smoke tests fail, the system is tagged as unstable, because it cannot ensure its basic functionality and it is not tested anymore until the smoke test pass. Smoke test are designed to uncover obvious errors which saves time, money and effort of the engineers. These tests should be used with every new build, since new features could harm previous system functionality.
The following lists show common scenarios for smoke testing:
\begin{itemize}
	\setlength\itemsep{0em}
	\item New system's build or version is ready for release.
	\item \todo{Anything else?}
\end{itemize}
Typical system issues covered by smoke testing testing:
\begin{itemize}
	\setlength\itemsep{0em}
	\item System without main functionality is useless.
	\item Main functionality can result into system crash.%%
\end{itemize} 


\subsection*{Regression Testing}
Whenever engineers develop new feature and want to update the previous build it has to pass the \emph{regression tests} \cite{STF:REGRESSION}. Regression tests are designed to test functionality of the latest build updated with new feature. The main objective is to determine, if new feature affects already functional parts of the system. This type of tests is very important, because engineers do not always realize, which parts of the system will be indirectly affected. During regression testing, new test cases are not created, but previous test cases are automatic re-executed and analyzed. 
Typical scenarios for regression testing:
\begin{itemize}
	\setlength\itemsep{0em}
	\item New feature of system is ready for use.
\end{itemize}
Common issues covered by regression testing:
\begin{itemize}
	\setlength\itemsep{0em}
	\item New feature could adversely affect already working components of the system.
\end{itemize}


\subsection*{Benchmark Testing}
\emph{Benchmark testing} \cite{Aho:Benchmarking} is method, which collects performance data during the system run on different hardware machines. Collected data have significant value when we want smooth run of the system on an older hardware, hence we can discover performance issues under normal load. However, the system does not run smoothly on prepared hardware, only options is to run benchmark tests on different machines with different hardware and under different load.

\section{Performance Metrics}
\label{Performance Metrics}
% https://loadstorm.com/load-testing-metrics/ 2017/10/18

During the performance testing we can monitor a lot of metrics, which can have different importance based on the system's purpose. The following lists the most common metrics that are monitored during the network performance testing of all applications, no matter of developing language. 

In the testing systems, performance metrics are collected during long process of collecting, analyzing and reporting information regarding performance of whole system or individual component. This process can be different for each metric, since each metric needs different type of the system analysis.


\subsection{Throughput}
Throughput is a metric, which refers to the number of requests per second that the system can handle. \emph{Network throughput} is the rate of successful message deliveries over a communication channel. Throughput is usually measured after a warm-up period of time after the commencement of traffic, because it takes a while for workload to stabilize. We need stabilized workload for proper throughput measurement, because initial values could negatively affect the measurement. Throughput is measured by load testing; suitable strategy for measuring throughput is to continuously raise the load until response takes longer that acceptable threshold.

\subsection{Response Time and Latency}
Response time as issue was already mentioned in Subsection \ref{Response Time 1}; response time as metric is consists of two parts which are \emph{latency} and \emph{service time}.

\subsubsection*{Service Time}
Service time is the time it takes system to evaluate and send response ti user request. In particular, when user sends request for a web page to a server, it takes the server time to evaluate the request and send proper response back to the user, this is the service time. Measurement can be performed easily using stopwatch which starts at receive of request and stops after the response is send. Service time can be affected by any item which leads to a performance degradation as described in Subsection \ref{Performance Degradation}. 

\subsubsection*{Latency}
Second part of the response time is latency \cite{Broadwell:RPT, BHATT:PERF}, which represent a delay between sending the request on the client side and receiving it for evaluation on the server side. Hence latency is the common problem in the network systems such as data center, web server, etc., because request/response needs to travel over the physical medium between the client and the server. Client and server could be located on different continents, thus the message have to travel long distance and latency is increasing.  

\begin{figure}[H]
  \centering
  \includegraphics[width=15cm]{obrazky-figures/latency.pdf}
  \caption{Latency diagram with system's response time based on the date.}
  \label{fig:latency}
\end{figure}

\subsubsection*{Average and Percentile Response Time}
There are two common ways of measuring the response time \cite{Kopp:RPT}: Average (mean) response time is calculated as the sum of all measured times divided by the count of users requests. While this seems trivial, in many times, the average response time does not actually reflect the real response time of the system. How is that possible? In reality, most application have few heavy outlines such as very slow transactions. In the Figure \ref{fig:average_percentil_1} you can see few slow transactions which drag the average of the response time to the right. This naturally leads to an inaccurate specification of response time.


\begin{figure}[H]
  \centering
  \includegraphics[width=15cm]{obrazky-figures/average_median_1.pdf}
  \caption{Transactions response time with calculated average and median of response time. Average represent  inaccurate response time in this case which is higher than real one.}
  \label{fig:average_percentil_1}
\end{figure}

The better solution how to determine the actual response time is Percentile. The percentile is statistic method, which cut measured ordered values into hundredths and then characterize the value below which a given percentage of measurements in a group of particular measurements falls. In the Figure \ref{fig:average_percentil_1} you can see the \emph{median} value, which reflects more realistic value of the system response time. Median value is same such as the 50th percentile. In this case, there is no problem, because user will expect slower response time than it has. 

\begin{figure}[H]
  \centering
  \includegraphics[width=15cm]{obrazky-figures/average_median_2.pdf}
  \caption{Transactions response time with calculated average and median of response time. Average represent  inaccurate response time. Average says, that STU is faster than is in reality.}
  \label{fig:average_percentil_2}
\end{figure}

The Figure \ref{fig:average_percentil_2} shows different situation. Average response time seems better than median, which reflects the expectation of faster system response time than it has. In real systems, we usually use values of the 90th percentile and the 99th percentile. 90th percentile mean, that there is only 10 \% transactions slower then marked response time. In this case, a considerable percentage of transactions are very fast (first 50 percent), while the bulk of transactions are several times slower. Thus, calculated percentile gets more realistic values than average response time.

\subsection{Resource Usage}
Applications running at servers with long run-time competes over a limited amount of resource available for use. Thus makes resource usage another important metric, which needs to be monitored since not enough resources could shut down the whole system. Main resources for monitoring and utilization are:

\begin{description}
	\setlength\itemsep{0em}
	\item \textbf{CPU usage}\,---\,inappropriate usage of CPU could lead to performance degradation, because low priority processes may occupy CPU ahead of the higher priority processes.
	\item \textbf{Memory usage}\,---\,full consumption of memory could cause performance degradation.
	\item \textbf{Disk space}\,---\,for example when using storage disk as a database, there should be preventive measures to backup the data and free up disk space.
	\item \textbf{Operating System limits}\,---\,system's memory, and CPU capabilities. 
\end{description}


\subsection{Error Rate}
Error Rate is a metric, which commonly occurs in the network systems, especially under high load. During the communication between client and server there could be error caused by another network device (router, switch, hub, etc.) or signal disturb which can corrupt the data during the transfer. The Error Rate is the mathematical calculation that produces a percentage of problem requests compared to all requests. In the ideal system, there should be zero network error present, however, in reality is in-feasible. This usually leads to a performance degradation and low throughput, because damaged data need to be resent.
Error rate is a significant metric because it tells engineers how many requests failed at a particular point in time of performance testing. This metric is more evident when you can see the percentage of problem strongly increasing, hence you can detect problem easily.


\todo{Druha iterace}

\chapter{Messaging Performance Tool}
\label{Messaging Performance Tool}
% https://github.com/orpiske/msg-perf-tool
Performance of \emph{Message-Oriented Middleware} (MOM) \cite{CURRY:MOM} is one of the most critical elements of quality assurence for enterprise integration system. There are multiple messaging components developed in Red Hat such as clients, Message Broker, Message Router (Qpid-dispatch service) and stream-like message distributions tools. \todo{Kafka - free to disclose?} 

A Message Broker is an example of MOM. Its purpose is receive, store and distribute messages, which are sent and received by clients. The performance capabilities of a Message Broker are important for its users, because being able to handle a large amount of transactions in a timely manner is an important characteristic of MOM. Users usually uses MOM for message distribution in their own systems where is necessary to send and receive message in short time. For example in automated systems, where components communicates with each other by command exchange. Amount of exchanged command is dependent on system size. We want to get systems results as soon as possible and for that is important to ensure smooth and quick message exchange.

\emph{The Maestro} \cite{ORPISKE:MSGPT} is a testing system designed especially for testing the performance of MOM. \todo{ROzepsat}

On the Figure \ref{fig:msg_perf_tool} you can see architecture of Maestro. The whole Maestro is deployed as a cluster system on several machines. A typical Maestro deployment consist of one node for Maestro Broker, one or more for Senders, one or more for Receivers and the S.U.T. The Maestro system consists of several components:

\begin{description}
	\setlength\itemsep{0em}
	\item \textbf{Maestro Broker}\,---\,any MQTT-capable broker with several topics. This component takes care about distribution of control messages between other cluster components such as Maestro Clients and MPT Back-end. Also, sending collected data to data server where user can browse between logs and show collected data as charts. 
	\item \textbf{Meastro Clients}\,---\,this component contains the client API as well as the test scripts for each test case. A sub-component called Reporter takes care of data reporting to the user, which means data visualization on the web.
	\item \textbf{MPT Back-end}\,---\,consists of sender part, receiver part and inspector part. Sender and receiver ensure messaging sending to the S.U.T and receiving from it. Inspector monitoring workload over the S.U.T and reporting collected performance metrics to the testing cluster. Maestro has two back-ends:
	\begin{itemize}
		\item \textbf{Java}\,---\,using for JMS-based testing, including AMQP, OpenWire and Core protocols.
		\item \textbf{C}\,---\,using for AMQP and STOMP protocol testing.
	\end{itemize}
\end{description}

\begin{figure}[H]
  \centering
  \includegraphics[width=15cm]{obrazky-figures/msg_perf_tool.pdf}
  \caption{Architecture of the Maestro.}
  \label{fig:msg_perf_tool}
\end{figure}


\section*{Test Case Configuration}
Configuration of each test case is specified by several options defined in Groovy \footnote{Object-oriented programming language for Java platform \url{http://groovy-lang.org/}} script. This script specify test behavior with following items:

\begin{itemize}
	\item \textbf{message size}\,---\,message size in bytes,
	\item \textbf{number of connected clients}\,---\,count of senders and receivers connected to the S.U.T,
	\item \textbf{test duration (time or load)}\,---\,end condition of each test, could be specified by time or message count,
	\item \textbf{message rate}\,---\,the desired rate that the system should try to maintain (0 for unbounded).
\end{itemize}

The test script is also responsible for starting and stopping the test. We can also specify the test profile. Depending on the test profile, the script may also be responsible for increasing or decreasing the workload on the S.U.T during the test scenario. The load can be modified by increasing the target rate or the number of parallel connections. Multiple combinations of this options can create a lot of test cases with different load for the S.U.T. Every test will produce his own logs which are processed by the reporting sub-component on the client side and used for monitoring the metrics. Maestro Reporter produce data visualization, such as charts, from stored logs.

\section*{Measures Process}
\label{Measures Process}
Measures process starts after dynamic test generation with option passed from configuration files. Sender or senders, depends on configuration file, will start sending messages to the S.U.T. Inspector starts monitoring the behavior of the S.U.T and report logs from S.U.T to Maestro Broker. However, there are two problem factors; the first is that inspector cannot influence performance of the S.U.T \todo{Email} The second is large size of the stored logs. This leads to use some of the compression methods for reduction logs size. However, compressed logs can fill whole hard drive during long test-run...\todo{Automatic collecting and archivation by the front-end. size: 1Gb uncompressed per hour of testing}


\section{Testing Metrics}
\label{Testing Metrics}
Throughput, latency, memory usage  \todo{Do you monitoring anything else?}

\section{Collected Data and Their Evaluation}
\label{Collected Data and Their Evaluation}



\section{Related Works}
\label{Related Works}
% popsat podobne "existujici" reseni (samozrejme, obcas neexistuje ;), ale verim, ze zde se neco najde). Nejlepsi je i se vuci tem "related tools" vymezit (jako napr. "The tool ... cannot be used because it does not support ...")

SpecJMS

\chapter{Analysis and Design}
\label{Analysis and Design}

\section{Qpid-Dispatch Router}

\section{Usable Protocols}
AMQP, MQTT - possibly?

\section{Automatic Topology Generator}

\subsection{Network Components}

\subsection{Structure of Input and Output}

\subsection{Topology Creation}

\section{Qpid-Dispatch Performance Module}

\subsection{\todo{more subsections about module}}

\section{Performance and Testing Metrics of Qpid-Dispatch Performance Module}

\section{Collected Data Evaluation}

\chapter{Implementation}
\label{Implementation}

\section{Used Technologies}

\subsection{Ansible}

\subsection{Docker}
Using for testing Ansible roles (remove?)

\section{Topology Generator}

\subsection{Template Generator}

\subsection{Generation of Variables}

\subsection{Configuration Files Generation and Deployment}

\section{Qpid-Dispatch Performance Module}

\subsection{TODO - more subsections about implementation}

\chapter{Experimental Evaluation}
\label{Experimental Evaluation}

\section{Performance Testing on Various Generated Topology}

\section{Testing results}

\chapter{Future work and ideas}
\label{Future work and ideas}

\chapter{Summary}
\label{Conclusion}
%=========================================================================
